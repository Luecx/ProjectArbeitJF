\chapter*{Kurzfassung} 
\markboth{Kurzfassung}{Kurzfassung}
%% ==============================


Die Schadensbeurteilung von Platten auf die ein Objekt gefallen ist oder ein Stoß verübt wurde, stellt eine wichtige Aufgabe in der Industrie dar. Jedoch werden in den meisten Arbeiten lediglich die Auswirkungen eines Stoßes auf eine Platte in der Plattenmitte analysiert. Jedoch hängt die Position des Stoßes auf die Platte eng mit der Schadensbeurteilung zusammen. Im folgenden wird eine Parameterstudie durchgeführt, welche Einwirkungen wie die Position, Gewicht, Geschwindigkeit und die Platteneigenschaften kombiniert. Zunächst wird festgestellt, unter welchen Bedingungen die Berechnungen gültig und zuverlässlich sind. Im Anschluss wird die eigentliche Parameterstudie anhand eines Programms durchgeführt welches die zuvor dargestellten mathematischen Grundlagen verwendet.

\begin{keywords}
Ca. Stoßvorgang, Biegung, Finite Elemente Methode, isotrope Materialen, Platten
\end{keywords}
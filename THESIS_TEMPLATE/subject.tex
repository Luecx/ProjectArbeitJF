%{\footnotesize \begin{center}-- Diese Titelseite ist nicht die Originaltitelseite zur Abgabe der Arbeit --\end{center}}
%---------------------------------------------------------------------------------
% Institutskopf
%---------------------------------------------------------------------------------
%\begin{tabularx}{\textwidth}{lXr}
%\parbox{100mm}{\centering\bf LEHRSTUHL UND INSTITUT F{\"U}R LEICHTBAU\\ %
%	 			Rheinisch-Westf{\"a}lische Technische Hochschule Aachen\\ %
%	 			Universit{\"a}tsprofessor Dr.-Ing. K.-U. Schr\"{o}der}	&&%
%\parbox{35mm}{\small D-52062 Aachen\\ W{\"u}llnerstr. 7\\ Tel: 0241/80-96830\\ Fax:  0241/80-92230}
%\end{tabularx}
\begin{textblock*}{110mm}(88mm,12mm)
\includegraphics*[height=16mm]{./pictures/rwth_sla}
\end{textblock*}
%
%---------------------------------------------------------------------------------
% Titel
%---------------------------------------------------------------------------------
\begin{textblock*}{\textwidth}(30mm,60mm)
\centering
{\bfseries Projektarbeit} \\[1ex]
\bfseries\scshape Studie zur Untersuchung des Verhaltens von Low-Velocity Impacts unter verschiedenen Parametern\par
\end{textblock*}
\vspace*{40mm}
Low-Velocity Impacts treten dort auf, wo Massen mit moderater Geschwindigkeit ein Ziel treffen. Hierzu gehören unter anderem Steinschläge oder fallen gelassene Werkzeuge auf Platten sowie Vogelschläge bei Flugzeugen. Je nach Randbedingung wie Plattengröße und Verhältnis der Massen können gänzlich unterschiedliche Verhalten während des Impakts beobachtet werden. Hierbei werden vornehmlich die zwei Fälle large mass impact und small mass impact unterschieden. Der erste Fall ist dadurch gekennzeichnet, dass die Kraft zwischen Impaktor und Platte sowie die Durchbiegung der Platte in Phase sind, während im zweiten Fall dies nicht gegeben ist. \\
Mehrere Parameter während eines solchen Impakts können das Verhalten stark beeinflussen. In dieser Projektarbeit sollen die Einflüsse dieser Parameter untersucht und quantifiziert werden. Grundlage hierfür bildet ein schon geschriebenes Python-Skript zur Berechnung eines Low-Velocity Impakts. Zunächst soll erarbeitet werden, unter welchen physikalischen Randbedingungen das Python-Skript die Realität korrekt beschreibt. Anschließend soll die eigentliche Parameterstudie durchgeführt werden. Hierbei soll zur Unterscheidung der oben genannten Fälle ein Kriterium erarbeitet werden. Die Studie endet mit einer Analyse der Ergebnisse.\\
Zur Steigerung der Wertigkeit der Projektarbeit können fakultativ weitere Untersuchungen durchgeführt werden. Diese beinhalten z.B. die Berücksichtigung von Vibrationen sowie der Einfluss von Membran- oder Querkräften in der Plattentheorie.\\

	


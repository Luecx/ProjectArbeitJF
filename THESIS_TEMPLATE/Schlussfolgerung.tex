\chapter{Schlussfolgerung und Ausblick}
\label{Conclusion}
%Was wurde in dieser Arbeit gemacht? Was sind die wesentlichen Ergebnisse? Wo ist weiterer Forschungsbedarf? Welche interessanten Forschungsbereiche ergeben sich aus der eigenen Arbeit?

In dieser Arbeit wurde eine Parameterstudie zur Untersuchung von Stößen auf isotropen Platten durchgeführt. Im Rahmen der Studie wurden die Parameter in Tabelle~\ref{tab:VariierteParams} ausgehend vom Ausgangsfall in Tabelle~\ref{tab:Ausgang} variiert.

\begin{table}[H]
	\begin{center}
		\caption{Parameterstudie}
		\label{tab:VariierteParams}
		\begin{tabular}{l|c}
			\textbf{Variable} & \textbf{Wert}\\
			\hline
			$h$ & Höhe der Platte [cm]\\
			$v_{0}$ & Geschwindigkeit der Impaktors [cm/s]\\
			$r_{g}$ & Radius des Impaktors [cm]\\
			$\frac{\xi}{a}$ & Auftreffstelle in x-Richtung [entdimensioniert]\\
			$\frac{\eta}{b}$ & Auftreffstelle in y-Richtung [entdimensioniert]\\
			$Sv$ & $\frac{a}{b} \equiv \; \mbox{Seitenverhältnis der Platte}$ \\		
		\end{tabular}
	\end{center}
\end{table}

Um Trends zu erkennen wurden unter anderem Faktoren betrachtet, die bei Kraft und Auslenkung bei Maximal- und Minimalwerten der Parameter auftreten.\\

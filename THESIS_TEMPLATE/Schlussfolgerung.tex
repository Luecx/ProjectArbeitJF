\chapter{Schlussfolgerung und Ausblick}
\label{Conclusion}
%Was wurde in dieser Arbeit gemacht? Was sind die wesentlichen Ergebnisse? Wo ist weiterer Forschungsbedarf? Welche interessanten Forschungsbereiche ergeben sich aus der eigenen Arbeit?

In dieser Arbeit wurde eine Parameterstudie zur Untersuchung von Stößen auf isotropen Platten durchgeführt. Zuerst wurde die benötigte Theorie in Kapitel~\ref{chap:Principles} hergeleitet. Anschließend wurden im Rahmen der Studie die Parameter in Tabelle~\ref{tab:VariierteParams} ausgehend vom Ausgangsfall in Tabelle~\ref{tab:Ausgang} variiert. In Kapitel~\ref{chap:Implementation} wurden die Veränderungen im Skript dargestellt und die Funktionsweise des Programmes erklärt. 

\begin{table}[H]
	\begin{center}
		\caption{Parameterstudie}
		\label{tab:VariierteParams}
		\begin{tabular}{l|c}
			\textbf{Variable} & \textbf{Wert}\\
			\hline
			$h$ & Höhe der Platte [cm]\\
			$v_{0}$ & Geschwindigkeit der Impaktors [cm/s]\\
			$r_{g}$ & Radius des Impaktors [cm]\\
			$\frac{\xi}{a}$ & Auftreffstelle in x-Richtung [entdimensioniert]\\
			$\frac{\eta}{b}$ & Auftreffstelle in y-Richtung [entdimensioniert]\\
			$Sv$ & $\frac{a}{b} \equiv \; \mbox{Seitenverhältnis der Platte}$ \\		
		\end{tabular}
	\end{center}
\end{table}

Wie in Kaptitel~\ref{chap:Durchfuehrung} ausgeführt, lassen sich einige Schlussfolgerungen aus den gewonnenen Ergebnissen ziehen. \\

\section{Anzahl der Aufschläge}
\label{sec:Aufschlag}

Interessant ist, dass die Anzahl der Aufschläge unter Einfluss der Auftreffgeschwindigkeit $v_{0}$ und des Radius' der Kugel $r_{g}$ nur von dem Massenverhältnis $Mr$ abhängt. Betrachtet man die Aufschläge unter Variation der Höhe, des Aufschlagortes oder des Seitenverhältnisses, haben die genannten Parameter einen messbaren Einfluss auf die Anzahl der Schläge. \\
Da die größte Kraft nicht immer beim Erstschlag auftritt, liegt es nahe, dass die Anzahl der Aufschläge und die mit ihnen verbundene Kraft für die Schadensbeurteilung wichtig sind.\\
%@FINN: Vielleciht hier noch was dazu über die Anzahl der Aufschläge bei xi und eta

\section{Auslenkung der Platte}
\label{sec:Auslenkung}

Bei der Auslenkung spielt die Höhe der Platte eine zentrale Rolle. Vergrößert man $h$ während $Mr$ konstant gehalten wird, fällt die Auslenkung $w$ schnell ab. Allerdings ist $w$ bei geringem $h$ sehr groß, was durch induzierte Spannungen in der Platte Schäden hervorrufen kann.\\
Betrachtet man die Geschwindigkeit des Impaktors, wird schnell ersichtlich, dass hier $v_{0}$ der dominierende Parameter ist. Zwar steigt die Auslenkung mit größer werdendem $Mr$ an, jedoch ist diese Änderung wesentlich kleiner als der Anstieg bei konstantem $Mr$ und größer werdender Geschwindigkeit.\\
Der Radius der Kugel hat nur einen verschwindend geringen Einfluss auf die Auslenkung der Platte. Wie in Kapitel~\ref{chap:Durchfuehrung}, Abbildung~\ref{fig:RadiusAuslenkung} zu sehen ist, steigert sich die Auslenkung von $r_{g,min}$ zu $r_{g,max}$ nur in der zweiten Nachkommastelle. \\


\section{Maximale Kraft}
\label{sec:Kraft}


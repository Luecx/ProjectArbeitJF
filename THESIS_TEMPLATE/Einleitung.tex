\chapter{Einleitung}
\label{chap:Intro}
%Die Einleitung liefert eine kurze Einordnung in den Projekthintergrund, um dem Leser den Einstieg in die Problemstellung zu ermöglichen. Sie kann je nach Art der Arbeit (Literaturrecherche, Konstruktion, praktische Arbeit, etc.) mehr oder weniger ausführlich sein, als Richtwert gilt ca. ein bis zwei Seiten.

%Zu den Inhalten der Einleitung zählen:
%
%\minisec{Zielsetzung}
%An dieser Stelle erfolgt die Beschreibung der Aufgabenstellung, Motivation und Zielsetzung. Hier wird zunächst das zugrunde liegende Problem bzw. die zugrunde liegende Beobachtung erläutert. Anschließend wird auf die wichtigste Fragestellung bzw. die eigentliche Forschungsfrage eingegangen, woraus sich letztendlich das Ziel und die praktischen Bedeutung der Arbeit ableitet.
%
%\minisec{Stand der Technik}
%Hier wird kurz darauf eingegangen, wo die Aufgabenstellung im aktuellen Stand der Technik einzuordnen ist. Dabei werden sowohl bereits geleistete Vorarbeiten als auch vergleichbare Tätigkeiten referenziert. Zusätzlich wird hier deutlich gemacht, was den Neuheitsgrad dieser Arbeit ausmacht.
%In diesem Kapitel wird zudem kurz umrissen, welche Methoden bisher bekannt sind, üblicherweise genutzt werden und inwieweit diese für die gegebene Aufgabenstellung von Relevanz sind. Dabei kann bereits eine Bewertung hinsichtlich der Zielsetzung erfolgen.
%
%\minisec{Übersicht}
%Zusätzlich soll die Einleitung eine kurze Übersicht über Aufbau und Gliederung der Arbeit geben in der kurz beschrieben wird, was in den einzelnen Kapiteln bearbeitet wird. Dies dient der Veranschaulichung des Roten Fadens dieser Arbeit.

%Die einzelnen Unterpunkte können in einem Fließtext zusammengefasst werden.

Schäden, die durch Stöße verursacht werden sind in vielen technischen Anwendungen von großer Bedeutung. Vor allem in der Luftfahrt, aber auch in anderen Bereichen, ist es für die Schadensbeurteilung und Wartungsplanung wichtig, zu wissen ob Panele und andere zum Teil tragende oder für die Funktion der Maschine essentielle Komponenten durch einen Stoß beschädigt wurden und ausgetauscht werden müssen. \\
Im Bereich der Low-Velocity Impacts, welche in dieser Arbeit untersucht werden, sind die äquivalenten Fälle zum Beispiel ein Werkzeug, das während Wartungsarbeiten auf den Flügel gefallen ist oder ein Bird-Strike. \\
Hier ist der vorwiegende Parameter die Masse des Stoßobjekts. Es wird vor allem zwischen High-Mass (hoher Masse) und Low-Mass (niedriger Masse) unterschieden. 
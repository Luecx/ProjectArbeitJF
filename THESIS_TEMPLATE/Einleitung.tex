\chapter{Einleitung}
\label{chap:Intro}
%Die Einleitung liefert eine kurze Einordnung in den Projekthintergrund, um dem
%Leser den Einstieg in die Problemstellung zu ermöglichen. Sie kann je nach Art
%der Arbeit (Literaturrecherche, Konstruktion, praktische Arbeit, etc.) mehr
%oder weniger ausführlich sein, als Richtwert gilt ca. ein bis zwei Seiten.

%Zu den Inhalten der Einleitung zählen:
%
%\minisec{Zielsetzung}
%An dieser Stelle erfolgt die Beschreibung der Aufgabenstellung, Motivation und
%Zielsetzung. Hier wird zunächst das zugrunde liegende Problem bzw. die zugrunde
%liegende Beobachtung erläutert. Anschließend wird auf die wichtigste
%Fragestellung bzw. die eigentliche Forschungsfrage eingegangen, woraus sich
%letztendlich das Ziel und die praktischen Bedeutung der Arbeit ableitet.
%
%\minisec{Stand der Technik}
%Hier wird kurz darauf eingegangen, wo die Aufgabenstellung im aktuellen Stand
%der Technik einzuordnen ist. Dabei werden sowohl bereits geleistete Vorarbeiten
%als auch vergleichbare Tätigkeiten referenziert. Zusätzlich wird hier deutlich
%gemacht, was den Neuheitsgrad dieser Arbeit ausmacht.
%In diesem Kapitel wird zudem kurz umrissen, welche Methoden bisher bekannt
%sind, üblicherweise genutzt werden und inwieweit diese für die gegebene
%Aufgabenstellung von Relevanz sind. Dabei kann bereits eine Bewertung
%hinsichtlich der Zielsetzung erfolgen.
%
%\minisec{Übersicht}
%Zusätzlich soll die Einleitung eine kurze Übersicht über Aufbau und Gliederung
%der Arbeit geben in der kurz beschrieben wird, was in den einzelnen Kapiteln
%bearbeitet wird. Dies dient der Veranschaulichung des Roten Fadens dieser
%Arbeit.

%Die einzelnen Unterpunkte können in einem Fließtext zusammengefasst werden.

Schäden, die durch Stöße verursacht werden, sind in vielen technischen
Anwendungen von großer Bedeutung. Vor allem in der Luftfahrt, aber auch in
anderen Bereichen, ist es für die Schadensbeurteilung und Wartungsplanung
wichtig zu wissen, ob Panels und andere zum Teil tragende oder für die Funktion
der Maschine essenzielle Komponenten, durch einen Stoß beschädigt wurden und
ausgetauscht werden müssen. \\
Im Bereich der Low-Velocity Impacts, welche in dieser Arbeit untersucht werden,
sind die äquivalenten Fälle zum Beispiel ein Werkzeug, das während
Wartungsarbeiten auf den Flügel gefallen ist oder ein Bird-Strike. \\
Der signifikante Parameter ist die Masse des Stoßobjekts. Hier wird vor allem
zwischen High-Mass (hoher Masse) und Low-Mass (niedriger Masse) unterschieden.
(CITATION NEEDED)\\
\\
Da die Masse hier eine zentrale Rolle spielt, wollen wir genauer auf die
Kriterien eingehen: \\
Laut der Gleichung für die kinetische Energie 
\begin{equation}
E = \frac{1}{2}mv^2
\end{equation}
,wobei 
\begin{table}[h!]
	\begin{center}
		\caption{Variablen}
		\label{tab:Tabelle 1}
		\begin{tabular}{l|c}
			\textbf{Variable} & \textbf{Bedeutung}\\
			\hline
			E & Kinetische Energie in Joule\\
			m & Masse in kg\\
			v & Geschwindigkeit in m/s\\
		\end{tabular}
	\end{center}
\end{table}\\
,geht die Masse nur mit der ersten Potenz, die Geschwindigkeit jedoch mit
der zweiten Potenz in die Energie ein. Hieraus ist leicht zu erkennen, dass die
Geschwindigkeiten im Low-Velocity Impact Bereich eine gegenüber der Masse
insignifikante Rolle spielt. \\
In der durchgeführten Parameterstudie werden unter anderem der Einfluss der
Stoßstelle, des Gewichts der Stoßmasse und des Materials sowie der Dimensionen der
Platte untersucht. \\
Für das Verständnis wird im Folgenden zunächst der aktuelle Stand der
Technik ausgeführt, bevor die physikalischen und mathematischen Grundlagen und
Modelle, auf denen das Skript basiert, erläutert werden.\\
%(Rest der Einleitung dann wenn die Arbeit soweit steht)
\\
\section{Zielsetzung}
Da in vielen Bereichen des Transportsektors kritische Komponenten Stößen ausgesetzt werden, liegt ein Bedarf zur Berechnung dieser Stöße und deren Folgeschäden vor. \\
In dieser Arbeit wird anhand der Berechnungsgrundlagen von Karas und eines
daraus abgeleiteten Python-Skriptes, eine Parameterstudie zur Untersuchung von
Low-Velocity Stößen auf isotrope Platten durchgeführt. \\
Das Ziel dieser Studie ist es, Zusammenhänge zwischen Parametern wie dem Stoßort,
der Stoßmasse, den Plattenmaßen und dem Plattenverhalten herauszuarbeiten. Diese
können dann für die Bestimmung von Schäden und Reparatur-, bzw.
Austauscharbeiten genutzt werden.\\
%hier noch mehr sobald der Hauptteil steht

\section{Stand der Technik}

%Stöße auf Platten haben in der Schadensbeurteilung seit langem eine stetig wachsende
%Wichtigkeit. \\
Bereits 1881 stellte Heinrich Hertz mit seiner Thesis über die Berührung fester
Körper die ersten Grundlagen für die Berechnung von Stößen und deren Folgen vor.
[HERTZ] \\
In seinen Überlegungen postuliert er, dass wenn ein gerundeter Körper auf eine
Platte trifft, sich dieser verformen muss, da sonst ein Unstetigkeitspunkt
auftritt. In diesem Punkt würde die Spannung ins Unendliche wachsen, denn laut
der Gleichung für Spannung 

\begin{equation}
\sigma = \frac{\mbox{F}}{\mbox{A}} \rightarrow \infty \; , \; \; \mbox{für die Fläche} \rightarrow 0
\end{equation}

würde die Fläche, wenn keine Verformung stattfindet, gegen null gehen. Die Drücke
und Deformationen in diesem Hertzschen Kontakt sind verglichen zu jenen
außerhalb der Kontaktzone undendlich groß. \\
Zur Vereinfachung der Berechnung stellt Hertz folgende Annahmen auf, die sich in
vielen späteren Thesen zu diesem Thema wiederfinden: 
\begin{itemize}
	\item Der übertragende Druck zwischen den Körpern ist endlich
	\item Die Kontaktflächen beider Körper sind ideal glatt
	\item Die Körper sind beide isotrop und elastisch
	\item Es stellt sich ein strikt senkrechter Druck zwischen den berührenden Teilen ein
\end{itemize}

In diesem Modell werden die beiden Flächen mathematisch in Berührung gebracht und das verwendete Koordinatensystem in die Fläche, die tangential zur Berührungsfläche liegt, eingebracht. \\

\begin{figure}[h!]
	\begin{center}
		\begin{overpic}[width=\linewidth]{pictures/einleitung/HertzscherKontakt.eps}
			\put(70,20){\mbox{x}}
			\put(51,1){\mbox{z}}
			\put(5,40){\begin{minipage}{\textwidth}Hertz'sche\\Flächenpressung\end{minipage}}
		\end{overpic}	
		\caption{Koordinatensystem und Flächenpressung nach Hertz}
		\label{fig:Hertz}
	\end{center}
\end{figure}

Basierend auf den Hertz'schen Berechnungen hat sich am Anfang des 20. Jahrhunderts auch Karas [KARAS] mit der Plattentheorie befasst. Auf seinen Berechnungen basiert in großen Teilen diese Arbeit. \\
Fast gleichzeitig stellte auch Timoshenko Überlegungen zu der Biegung einer gestützten Platte unter Wirkung einer Einzellast an. Basierend auf der Arbeit von Dr. Nadai leitet er anhand unendlich ausgedehnter Platten einen tabellarischen Ansatz zur Lösung des oben genannten Problems her. Auch zeigte sich in seinen Berechnungen, dass die Werte für eine endliche, eingespannte Platte mit stetig größer werdenden Maßen, sich denen der unendlich ausgedehnten Platte schnell nähern.\\
Auch Wang [WANG et al.] stellten Überlegungen zu dünnen Platten auf. In ihrer These NONLINEAR LARGE-DEFLECTION BOUNDARY-VALUE PROBLEMS OF RECTANGULAR PLATES von 1948 beschreiben sie die Auslenkung, die solche Platten unter seitlichem Schub oder normalem Druck erfahren. Bei kleinen Auslenkungen unterliegt die Deformation einer einfachen linearen Gleichung. Sobald die Auslenkung groß wird, fällt die Beschreibung unter die sogenannten Von Kármán'schen Differentialgleichungen, die Wang et al. durch Finite Differenzen lösen.\\
In der zweiten Hälfte des 20. Jahrhunderts richtet sich der Fokus der Forschung auf Laminatplatten. Shivakumar et al. [SHIV] stellen zwei Modelle zur Bestimmung der Einschlagkraft und Dauer bei Low-Velocity Impactversuchen in zirkularen Laminaten auf. \\
Hierbei leiten sie ein Feder-Masse Modell und ein Energiemodell her. \\
Das Energiemodell basiert auf der Erhaltung der Energie. In einer einfachen Gleichung wird die kinetische Energie des Stoßkörpers mit der Summe der Kontakt-, Deformations- und Schubspannungsenergien gleichgesetzt. Diese Gleichung wurde dann nach der Stoßkraft umgestellt und durch die Newton-Raphson Methode gelöst.

\begin{equation}
	\frac{1}{2} m_{g} V_{0}^{2} = E_{Kontakt} + E_{Biege} + E_{Deform}
\end{equation}

Das Feder-Masse Modell stellen die Autoren als Weiterführung des Modells nach Lee auf. Die Stoßmasse $m_g$ und die Plattenmasse $m_p$ werden hier als rigide Massen dargestellt. In früheren Versuchen wurde gezeigt, dass die effektive Plattenmasse ungefähr $\frac{1}{4}$ der Gesamtmasse beträgt. Deshalb wird hier 

\begin{equation}
m_{eff} = \frac{1}{4} m_p
\end{equation} 

angenommen. Die resultierenden Gleichungen, mit zwei Freiheitsgraden, setzen sich wie folgt zusammen:

\begin{figure}[h!]
	\begin{center}
		\begin{overpic}[scale=2.2]{pictures/einleitung/SpringMassModel.eps}
			\put(-5,73){$x_{1}$(t)}
			\put(-5,38){$x_{2}$(t)}
			\put(20.5,76){$m_{g}$}
			\put(20.5,41.5){$m_{p}$}
			\put(30,58){Feder, c}
			\put(-2,25.5){$K_{Biege}$}
			\put(-2,10.5){$K_{Feder}$}
			\put(38,19){$K_{Deform}$}
			\put(80,73){$m_{g}$$\ddot{x_{1}}$}
			\put(80,28){$m_{p}$$\ddot{x_{2}}$}
			\put(97,54){$c\cdot(x_{1}-x_{2})^\frac{3}{2}$}
			\put(97,6){$K_{B,F}x_{2}+K_{D}\cdot x_{2}^{3}$}
		\end{overpic}
		\caption{Feder-Masse Modell mit zwei Freiheitsgraden}
		\label{fig:SPM}
	\end{center}
\end{figure}

\begin{equation}
	m_{g} \ddot{x_{1}} + \lambda c \cdot \bigl| (x_{1} - x_{2}) \bigl| ^{\frac{3}{2}} = 0
\end{equation}

für die Stoßmasse und

\begin{equation}
	m_{p} \ddot{x_{2}} + K_{B,F} x_{2} + K_{D} x_{2}^{3} - \lambda c \cdot \bigl| (x_{1} - x_{2}) \bigl| ^\frac{3}{2} = 0
\end{equation}

für die Platte, mit:

\begin{align}
		\lambda &= 1 \quad \mbox{für} \quad x_{1} > x_{2} \\
		\lambda &= -1 \quad \mbox{für} \quad x_{1} < x_{2}
\end{align}


Die beiden Gleichungen werden hier durch das Mehrschrittverfahren gelöst. 

Beide Modelle beinhalten hierbei sowohl die Plattendeformation, als auch die Impaktordeformation. Das Feder-Masse Modell erwies sich allerdings als praktischere Lösung, da sie neben der maximal auftretenden Kraft auch den Kraftverlauf und die Auslenkung der Platte wiedergeben kann. \\
Auch Olsson befasst sich mit Impact-Versuchen auf Platten. Allerdings erweitert er die bestehende Theorie mit einem analytischen Modell, um Schäden in Laminatplatten, also anisotropen Platten, die durch Stößen großer Massen verursacht werden. In seinem Aufsatz stellt er außerdem ein Massenkriterion für das Massenverhältnis von Impactor zu Platte auf. [Olsson]\\
Wichtig ist hier, dass bei sehr kleinen Impaktmassen, die Platte ballistisch reagiert. Dieses Verhalten wird durch Druckwellen durch die gesamte Platte dominiert. Weiterhin unterscheidet Olsson zwischen Small Impact-Mass (kleiner Stoßmasse) und Large Impact-Mass (großer Stoßmasse). Das Plattenverhalten bei diesen kleinen Stoßmassen wird vor allem durch Scher- und Biegewellen, die nicht in Phase laufen, beschrieben. Bei Stoßmassen, die wesentlich größer als die Plattenmasse sind, wird ein 'quasi-statisches' Verhalten beobachtet. Hier sind die Spannungen, Verformung und maximalen Belastungen ungefähr in Phase. Auch von Interesse ist, dass Stöße von kleineren Stoßkörpern eher Schäden verursachen als große Stoßkörper mit der gleichen kinetischen Energie.\\
Das Massenkriterion stellt Olsson wie folgt dar.\\
Für die kleinen Stoßmassen kombiniert Olsson die Gleichung für die Position der Vorderkante des n-ten Wellenmodus in Polarkoordinaten:

\begin{equation}
	r_{n}(\theta) = \sqrt{\pi}(\frac{2D_{r}(\theta)}{m})^{\frac{1}{4}} (D \cdot l\sqrt{D_{11}D_{22}})^{\frac{1}{4}}\sqrt{nt}
\end{equation}

wobei $D_{r}(\theta)$ die radiale Plattenbiegesteifigkeit in Richtung $\theta$ und $D_{ii}$ die Plattenfestigkeiten darstellen, mit der Gleichung für die Masse der Platte, die von dem Stoß beeinflusst wird:

\begin{equation}
	m_{p,Welle} = m \pi r_{1}(0^{\circ}) r_{1}(90^{\circ}) = \pi^{2} t_{imp} \sqrt{2mD^{*}}
\end{equation}

und die Gleichung für die größte Plattenmasse $m_{p,max}$, die von den Plattengrenzen unbeeinflusst bleibt. Diese findet man, indem man die kleinere der beiden elliptischen Flächen, die durch

\begin{equation}
	m_{p,max} = m \pi \cdot min\{(\frac{D_{22}}{D_{11}})^{\frac{1}{4}} r_{x,min}^{2} \; , \; (\frac{D_{11}}{D_{22}})^{\frac{1}{4}} r_{y,min}^{2} \}
\end{equation}		
		
beschrieben werden, betrachtet. 

\begin{table}[h!]
	\begin{center}
		\caption{Variablen}
		\label{tab:Tabelle 2}
		\begin{tabular}{l|c}
			\textbf{Variable} & \textbf{Bedeutung}\\
			\hline
			$t_{imp}$ & Zeit am Ende des Stoßvorgangs\\
			m & Stoßmasse in kg\\
			$r_{x,min}$ & Distanz vom Stoß bis zum nächsten Plattenrand in x-Richtung\\
			$r_{y,min}$ & Distanz vom Stoß bis zum nächsten Plattenrand in y-Richtung\\
		\end{tabular}
	\end{center}
\end{table}

Ein hinreichendes Kriterium für die wellen-kontrollierte Plattenantwort (kleine Stoßmasse) ist, dass die affektierte Plattenmasse bei $t_{imp}$ nicht größer ist als die Masse, die beeinflusst wird, wenn die Grenzen erstmals erreicht werden. Die kombinierten Gleichungen liefern dann: 
\begin{equation}
	\frac{m}{m_{p,max}} \leq \frac{m}{m_{p,Welle}} \leq 2 \frac{\sqrt{2}}{\pi^{2}} \approx 0.29
\end{equation}

Bei großen Stoßmassen geht Olsson vor, indem er ein Feder-Masse System modelliert, da die Stoßantwort der Platte, für genügend große Stoßzeiten, quasi-statisch ist. Quasi-statisch bedeutet hier, dass die Last und Auslenkung die gleiche Beziehung wie im statischen Fall haben. Durch Zusammenfügen der Stoßzeit $t_{imp}$ und der halben Periodendauer des ersten Vibrationsmodus der Platte ergibt sich: 

\begin{equation}
	\frac{m}{m_{p,max}} \gg \frac{1}{4}
\end{equation}

Zuletzt wird noch auf das Kriterion für mittlere Stoßmassen eingegangen. Bei einem zentrierten Stoß auf einee quasi-isotropische Platte ergibt sich:

\begin{equation}
	\frac{1}{5} < \frac{m}{m_{p,max}} < 2
\end{equation}

Da dieser jedoch für diese Arbeit nicht von großer Relevanz ist, wird an dieser Stelle nicht näher darauf eingegangen.\\

\begin{figure}[H]
	\begin{center}
		\begin{overpic}[width=\linewidth]{pictures/einleitung/MassenkriterionOlsson.eps}
					
		\end{overpic}	
		\caption{Massenkriterion nach Olsson}	
		\label{fig:Olsson}
	\end{center}
\end{figure}


Im 21. Jahrhundert nahmen dann Kolvik, Onate und Djojodihardjo die Forschung an Platten und Stößen auf.\\
Djojodihardjo und Mahmud [DJOJO] legen ihren Schwerpunkt auf die Simulation und Modellerstellung von Stößen auf Platten, vor allem im Rahmen der Raumfahrt. In ihren Überlegungen befassen sie sich mit den Stoßantwortcharakteristika von isotropen und homogenen Platten mit verschiedenen Eigenschaften, die zusammengefügt werden und dann von einem Impactor gestoßen werden. Ausgehend von der Mindlin-Plattentheorie erstellen Djojodihardjo et al. einen Algorithmus, um Platten als generische Strukturen unter Stoßlast zu analysieren. Weiterführend benutzen sie diesen Algorithmus dann für numerische Simulationen, um daraus eine optimale Designkonfiguration für solche Platten zu entwickeln.\\
Eine weitere Berechnungsmethode wurde unter Annahme der Modelle von Reissner und Mindlin von Onate entworfen [ONATE]. Diese ist auch gütlig, wenn die Kirchoff'sche Plattenbedingung nicht anwendbar ist:

\begin{equation}
\frac{\mbox{Plattendicke}}{\mbox{Seitenlänge}} = \frac{1}{10}
\end{equation}

In der sogenannten Reissner-Mindlin Plattentheorie wird angenommen, dass die Plattennormale unter Deformation nicht orthogonal zur Plattenmittelebene
bleibt. Hierdurch eignet sich diese auch zur Berechnung von Schubspannungseffekten.

\begin{figure}[H]
	\begin{center}
		\begin{overpic}[width=\linewidth]{pictures/einleitung/MindlinPlate.eps}
			\put(10,23){\begin{minipage}{\textwidth}Eigentliche\\ Normalendeformation\end{minipage}}
			\put(73,5){\begin{minipage}{\textwidth}Angenommene\\ Normalendeformation\end{minipage}}
		\end{overpic}	
		\caption{Normalendeformation nach Mindlin}
		\label{fig:mindlin}
	\end{center}
\end{figure}

Da die Berechnung nach Reissner-Mindlin das Lösen von Differentialgleichungen höherer Ordnung erfordert, stellt sie für diese Arbeit nur einen geringen Anteil
dar und wird daher bei Bedarf genauer erläutert.\\

Es ist offensichtlich, dass vor allem im Bereich der isotropen Platten noch Forschungsbedarf herrscht. Im weiteren Verlauf dieser Arbeit wird auf dieses Thema weiter eingegangen.\\





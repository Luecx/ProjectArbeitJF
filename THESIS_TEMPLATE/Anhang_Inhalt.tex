\chapter{Anhang - Tipps zum Verfassen studentischer Arbeiten}
\minisec{Allgemeine Regeln:}
\begin{compactitem}
\item[\emph{Neutrale Formulierung:}] Die Arbeit wird in der 3. Person Singular verfasst.
\item[\emph{Tempus:}] Die Arbeit wird durchgängig im Präsens geschrieben.
\item[\emph{Aktive Formulierung:}] Formulierungen im Passiv vermeiden.
\item[\emph{Formulierung im Indikativ:}] Die Verwendung des Konjunktivs vermeiden.
\item[\emph{Abkürzungen:}] Alle Abkürzungen, die nicht im Duden stehen, müssen eingeführt werden.
\end{compactitem}
\minisec{Generell gilt:}
\begin{itemize}
\item[\emph{So wenig wie möglich, so viel wie nötig!}] Herleitungen und Erklärungen auf das Wesentliche beschränken und nicht zu weit abschweifen; Vermeidung der Aufzählung unnötiger Einzelheiten, von allgemein Bekanntem und von Informationen, die nicht zum Thema gehören.
\item[\emph{So einfach wie möglich:}] Möglichst kurz und prägnant formulieren. 
\item[\emph{Kurze Sätze:}] Vermeidung von komplizierten Satzreihen und Schachtelsätzen sowie von langen zusammengesetzten Wörtern.
\item[\emph{Fachvokabular Sätze:}] An den richtigen Stellen einheitlich und durchgängig verwenden. Für das Verständnis notwendige, aber nicht geläufige Fach- und Fremdwörter sowie Abkürzungen erläutern.
\item[\emph{Redundanz:}] Unbekannte und besonders wichtige Informationen müssen dem Leser durch angemessene Wiederholung vors Auge geführt werden. 
\item[\emph{Übersichtlichkeit:}] Ergebnisse, Versuchspläne, Maschinendaten, etc. möglichst in Tabellenform abfassen und nicht im Fließtext aufzählen.
\item[\emph{Aussagekräftige Kapitelüberschriften}] (nicht zu lang!) erleichtern das Verständnis.
\item[\emph{Unterkapitel}] sind nur bis zur vierten Stufe zulässig; Eine neue Stufe ist nur dann einzuführen, wenn es mindestens zwei Kapitel in dieser Ebene gibt.
\item[\emph{Kapiteleinleitung und -abschluss:}] Ein bis zwei \emph{kurze Einleitungssätze} am Anfang von jedem Kapitel (z.B. In diesem Kapitel geht es um..., Im Folgenden wird beschrieben...) bzw. \emph{zusammenfassende Sätze} am Ende eines Kapitels geben dem Text eine Struktur und tragen so zum allgemeinen Verständnis bei.
\end{itemize}

Der Umfang der schriftlichen Arbeit richtet sich nach der jeweiligen Prüfungsordnung. Es gilt:
\begin{center}
\begin{tabular}{lr}
\toprule
{Projektarbeit} & k.A.\\
{Bachelorarbeit} & < 50 Seiten ohne Anhang\\
{Masterarbeit} & < 80 Seiten ohne Anhang\\
\bottomrule
\end{tabular}
\end{center}
%
\minisec{Kontrollfragen:}
\begin{itemize}
\item Ist der \emph{Rote Faden} durch die ganze Arbeit erkennbar?
\item Folgt der Aufbau der Arbeit einer \emph{logischen Struktur} (Reihenfolge, Gedankensprünge, Sinnabschnitte,etc.)?
\item Gibt es eine durchgängige \emph{Zeitform} (Präsens), oder gibt es Zeitsprünge (z.B. Präteritum im Durchführungsteil)?
\item Stimmen die \emph{inhaltlichen Bezüge}? (Passt die Verbform zum Subjekt?)
\item Sind die Inhalte \emph{übersichtlich} gestaltet?
\end{itemize}



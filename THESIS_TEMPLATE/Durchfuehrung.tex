\chapter{Durchführung}
\label{chap:Durchfuehrung}

In der bereits hergeleiteten Theorie wird der Lösungsweg für das beschrieben Problem vorgestellt. Mit dem zur Verfügung gestelltem Skript, das auf dieser Theorie basiert wird nun die Parameterstudie durchgeführt. Um relevante Ergebnisse zu erlangen, müssen zuerst weitere Gleichungen in das Skript eingearbeitet werden.\\
In dem Skript wird eine Konstante $c$ als Vorfaktor für die Kraftberechnung aus dem Punktkontakt verwendet. Um die Berechnungsmöglichkeiten zu erweitern, wird $c$ nun durch eine Gleichung der Hertz'schen Pressung ersetzt, die wie folgt Hergeleitet wird:\\
Die Gleichung für die Hertz'sche Pressung eines Kugel-Kugel-Punktkontaktes lautet wie in Gleichung~\ref{form:ZvonP}

\begin{equation}
	\label{form:ZvonP}
	z(P) = \left[ \frac{9 P^{2} (1 - \nu^{2})^{2}}{2 E^{2}} \cdot \left( \frac{1}{2 r_{1}} + \frac{1}{2 r_{2}} \right) \right]^{\frac{1}{3}}
\end{equation}

Da einer der beiden Körper eine Platte mit dem Radius $r_{p} = \infty$ ist, folgt durch einsetzen von $r_{p}$ und umstellen nach P:

\begin{equation}
	\label{form:PvonZ}
	P(z) = \sqrt{\frac{4 r_{g}}{9}} \cdot \frac{E}{(1 - \nu^{2})} \cdot z^{\frac{3}{2}}
\end{equation}
	
Hier ist $\frac{E}{(1 - \nu^{2})}$ der reduzierte E-Modul, welcher durch Gleichung~\ref{form:redE} beschrieben wird.

\begin{equation}
	\label{form:redE}
	\frac{E}{(1 - \nu^{2})} = \left( \frac{(1 - \nu_{p}^{2})}{E_{p}} + \frac{(1 - \nu_{g}^{2})}{E_{g}} \right)^{-1}
\end{equation} 

Nun wird $c$ als Vorfaktor von $z^{\frac{3}{2}}$ in Gleichung~\ref{form:PvonZ} definiert, sodass folgt:

\begin{equation}
	\label{form:c}
	c = \sqrt{\frac{4 r_{g}}{9}} \cdot \frac{E}{(1 - \nu^{2})}
\end{equation}

Im Skript wurden Gleichungen~\ref{form:redE} und~\ref{form:c} unter den Konstanten implementiert und die einstellbaren Parameter der Platte und der Impaktor respektive um $E_{p}$ und $\nu_{p}$, bzw. $E_{g}$ und $\nu_{g}$ erweitert.\\

\section{Vorgehen und Vorbereitung}

Insgesamt wird das Problem durch die einstellbaren Parameter in Tabelle~\ref{tab:VariablenderStudie} beschrieben. In Anlehnung an \cite{Olsson.2000} wird das Massenverhältnis $Mr$ als einer der signifikanten Parameter gewählt. Davon ausgehend werden dann die anderen Parameter einzeln variiert und die Ergebnisse grafisch ausgewertet. 

\begin{table}[H]
	\begin{center}
		\caption{Variablen der Parameterstudie}
		\label{tab:VariablenderStudie}
		\begin{tabular}{l|c}
			\textbf{Variable} & \textbf{Bedeutung}\\
			\hline
			$a,b,h$ & Plattenmaße in [cm]\\
			$r_{g}$ & Impaktorradius in [cm]\\
			$v_{0}$ & Auftreffgeschwindigkeit [cm/s]\\
			$Mr$ & Massenverhältnis $\hat{=}$ $\frac{Impaktormasse}{Plattenmasse}$\\
			$\xi,\eta$ & Auftreffstelle des Impaktors\\
			$x,y$ & Auswertungsstelle\\		
		\end{tabular}
	\end{center}
\end{table}

\subsection{Auswertungskriterien und Annahmen}

Als Auswertungskriterien werden die Anzahl der einzelnen Schläge und der maximal auftretenden Kraft zusammen mit der Auslenkung der Platte gewählt. \\
Für die Studie werden folgende übergreifende Annahmen getroffen: 

\begin{enumerate}
	\item{Die Platte und der Impaktor sind isotrop, mit konstanten Stoffwerten}
	\item{Die Platte und der Impaktor sind aus identischem Material, mit gleichem E-Modul und Poissonzahl $\nu$}
\end{enumerate}

\subsubsection{Anzahl der Schläge}

Als Schlag wird hier ein Kraftverlauf bezeichnet, bei dem die Kraft $P_{i}$ zwischen den Schlägen auf Null fällt, nach der Bedingung in Gleichung~\ref{form:Schlagauftreten}:

\begin{equation} 
	\label{form:Schlagauftreten}
	F_{i-1} > 0 \; \wedge \; F_{i} = 0.0 
\end{equation}

Da im zeitlichen Verlauf auch nach dem ersten Auftreffen noch Schläge auftreten können, wird die Schrittanzahl auf $i = 1000$ gesetzt um signifikante Ergebnisse zu erlangen. 

\subsubsection{Maximale Auslenkung der Platte}

Die maximale Auslenkung $w_{max}$ wird direkt aus der Versuchsreihe ausgelesen und in Abhängigkeit des Massenverhältnisses und des zu veränderten Parameters dargestellt.

\subsubsection{Maximale Kraft}

Analog zur maximalen Auslenkung, wird die maximale Kraft $P_{max}$ direkt ausgelesen und in Abhängigkeit von $Mr$ und dem Parameter dargestellt.



\section{Parameterstudie}

Für die Parameterstudie wurde folgender Fall aus Ausgangspunkt gewählt: 

\begin{table}[H]
	\begin{center}
		\caption{Ausgangsfall Parameterstudie}
		\label{tab:Ausgang}
		\begin{tabular}{l|c}
			\textbf{Variable} & \textbf{Wert}\\
			\hline
			$a$ & 50.0 [cm]\\
			$b$ & 50.0 [cm]\\
			$h$ & 1.0 [cm]\\
			$r_{g}$ & 1.0 [cm]\\
			$v_{0}$ & 500.0 [cm/s]\\
			$\xi,\eta$ & 25.0 [cm]\\
			$x,y$ & 25.0 [cm]\\ 		
		\end{tabular}
	\end{center}
\end{table}

$Mr$ wird hier bewusst ausgelassen, da bei jedem Parameter über das Massenverhältnis iteriert wird und dadurch kein Anfangswert, sondern eine Menge, benötigt wird. Die Menge der $Mr$ ist nach Gleichung~\ref{form:Mr}:

\begin{equation}
	\label{form:Mr}
	0.01 \leq Mr \leq 2.50 \; , \;\; \mbox{Schrittweite:} \; \Delta Mr = 0.01
\end{equation}

Die Stoffparameter sind im Rahmen der Studie konstant und in Tabelle~\ref{tab:Stoff} definiert.

\begin{table}[H]
	\begin{center}
		\caption{Stoffparameter: Kugel und Platte}
		\label{tab:Stoff}
		\begin{tabular}{l|c}
			\textbf{Variable} & \textbf{Wert}\\
			\hline
			$E_{p}$ & $2.2 \cdot 1e06$ [$kg/cm^2$]\\
			$\nu_{p}$ & 0.3 [-]\\
			$\rho_{p}$ & 0.00796 [$kg/cm^{3}$]\\
			\hline
			$E_{g}$ &  $2.2 \cdot 1e06$ [$kg/cm^2$]\\
			$\nu_{g}$ & 0.3 [-]\\		
		\end{tabular}
	\end{center}
\end{table}

\subsection{Höhe der Platte}

Als erster Parameter wird die Höhe nach Gleichung~\ref{form:DeltaH} variiert. Alle anderen Werte werden nach Tabelle~\ref{tab:Ausgang} konstant gehalten.

\begin{equation}
	\label{form:DeltaH}
	0.5 [cm] \leq h \leq 2.5 [cm], \; \; \mbox{Schrittweite:} \; \Delta h = 0.1 [cm]
\end{equation}

Wenn man die Höhe, das Massenverhältnis und die Anzahl der Schläge aufträgt, ergibt sich Abbildung~\ref{fig:Hoehe}.

\begin{figure}[H]
	\begin{center}
		\begin{overpic}[width=\linewidth]{pictures/gnuplot/3d/Hoehe/production/Hoehe.eps}
			\put(45,4){Höhe [cm]}
			\put(18,30){\rotatebox{90}{Mr [-]}}
			\put(83,60){Aufschläge [-]}
		\end{overpic}
	\caption{Höhe, Mr und Anzahl der Aufschläge}
	\label{fig:Hoehe}
	\end{center}
\end{figure}

Man kann direkt ablesen, dass mehrere Aufschläge vermehrt im unteren Höhenbereich auftreten. Dies ist direkt auf die Steifigkeit der Platte zurückzuführen. Bei größerer Plattendicke ist die Steifigkeit der Platte größer und die Kugel prallt ab, ohne vermehrt aufzutreffen. Bei $h = 0.6 [cm]$ treten bis zu 10 Schlägen auf. Interessant ist auch, dass sich drei Bereiche in Abhängigkeit des Massenverhältnisses ausbilden. \\
Es ist offensichtlich, dass bei einem Massenverhältnis $Mr \geq 1.7$ und einer Plattendicke von $h \geq 1.6 [cm]$ nur noch einzelne Stöße auftreten. Hier ist anzunehmen, dass der elastische Stoßvorgang der Kugel ausreichend kinetische Energie entgegen der Stoßrichtung überträgt, dass sie sich schneller entfernt, als die Platte auslenken kann. \\
Auch bei der Auslenkung stimmen die Ergebnisse mit den Erwartungen überein. In Abbildung~\ref{fig:HoeheAuslenkung} ist zu erkennen, dass mit zunehmender Höhe die Auslenkung abfällt, da die Steifigkeit der Platte zunimmt. Die Steigung der Auslenkungskurve flacht mit zunehmendem $Mr$ ab. Am signifikantestem ist die Steigung im Bereich $0.01 \leq Mr \leq 1$. Ergo ist der Einfluss des Massenverhältnisses am größten in diesem Bereich.\\
Bei Plattendicken von $h \geq 1.3 [cm]$ flacht die Steigung der maximalen Auslenkung in einen annähernd linearen Bereich ab. Daher liegt nahe, dass die Höhe im Bereich bis $h = 1.3 [cm]$ den stärksten Einfluss hat.\\

\begin{figure}[h!]
	\begin{center}
		\begin{overpic}[width=\linewidth]{pictures/gnuplot/3d/Hoehe/production/HoeheAuslenkung.eps}
			\put(60,9){Höhe [cm]}
			\put(12,14){Mr [-]}
			\put(1,58){Auslenkung [cm]}
		\end{overpic}
	\caption{Höhe, Mr und Auslenkung}
	\label{fig:HoeheAuslenkung}
	\end{center}
\end{figure}

Die größte Auslenkung tritt bei einem Massenverhältnis von $Mr = 2.5$ und einer Höhe von $h = 0.5 [cm]$ auf und beträgt $w(0.5,2.5) = 2.265 [cm]$, während die geringste Auslenkung bei $Mr = 0.01$ und $h = 2.5 [cm]$ mit $w = 0.0165 [cm]$ auftritt.\\
\\
Wenn man die maximal auftretende Kraft betrachtet, ergibt sich Abbildung~\ref{fig:HoeheKraft}.\\
Aus der maximalen Auslenkung kann man schlussfolgern, dass die maximale Kraft dort auftritt, wo die Auslenkung am geringsten ist, da hier am wenigsten Energie in die Deformation der Platte übergeht.\\
Bei $Mr = 2.5$ und $h = 2.5 [cm]$ tritt daher eine Kraft von $P = 598.36 [kN]$ auf. Die hier auftretende Kraft ist etwa um den Faktor $50$ größer als die geringste Kraft, die wie zu erwarten bei $Mr = 0.01$ und $h = 0.5 [cm]$ auftritt und $P = 10.88 [kN]$ beträgt. \\
Interessant zu betrachten ist auch, wie die Kraft mit größer werdendem $Mr$ zunimmt. Bei $h_{min} = 0.5 [cm]$ ist die Kraft bei $Mr_{max} = 2.5$ nur um den Faktor 2 größer als bei $Mr_{min} = 0.01$. Betrachtet man allerdings $h_{max} = 2.5 [cm]$, ist die Kraft bei $Mr_{max} = 2.5$ um den Faktor $14.4$ größer als bei $Mr_{min} = 0.01$, bzw. betragen die Kräfte $P_{max}(h_{max}) = 598.36[kN]$ und $P_{min}(h_{max}) = 41.4[kN]$.

\begin{figure}[H]
	\begin{center}
		\begin{overpic}[width=\linewidth]{pictures/gnuplot/3d/Hoehe/production/HoeheKraft.eps}
			\put(45,4){Höhe [cm]}
			\put(18,30){\rotatebox{90}{Mr [-]}}
			\put(80,62){Kraft [N]}
		\end{overpic}
	\caption{Höhe, Mr und Kraft}
	\label{fig:HoeheKraft}
	\end{center}
\end{figure}



\subsection{Geschwindigkeit des Impaktors}

Als nächster Parameter wird die Geschwindigkeit betrachtet. Da hier nur der Low-Velocity Bereich betrachtet wird, wird die Geschwindigkeit nach Gleichung~\ref{form:DeltaV0} variiert. 

\begin{equation}
	100 [cm/s] \leq v_{0} \leq 1000 [cm/s] \; , \;\; \mbox{Schrittweite:} \; \Delta v_{0} = 10 [cm/s]
	\label{form:DeltaV0}
\end{equation}

Analog zur Höhenvariation wird zuerst die Anzahl der Aufschläge aufgetragen. In Abbildung~\ref{fig:Speed} ist zu erkennen, dass %weiter wenn ne bessere Darstellung steht

\begin{figure}[h!]
	\begin{center}
		\begin{overpic}[width=\linewidth]{pictures/gnuplot/3d/Speed/production/Speed.eps}
			\put(40,4){Geschwindigkeit [cm/s]}
			\put(18,30){\rotatebox{90}{Mr [-]}}
			\put(83,60){Aufschläge [-]}
		\end{overpic}
		\caption{Geschwindigkeit, Mr und Anzahl der Aufschläge}
		\label{fig:Speed}
	\end{center}
\end{figure}

Die aus den Stößen resultierende Auslenkung wird in Abbildung~\ref{fig:SpeedAuslenkung} dargestellt. Man kann erkennen, dass der Einfluss des Massenverhältnisses mit zunehmender Geschwindigkeit zunimmt. \\
Die geringste Auslenkung tritt bei $Mr = 0.01$ und $v_{0} = 100 [cm/s]$ auf und beträgt $w = 0.008899 [cm]$.

\begin{figure}[H]
	\begin{center}
		\begin{overpic}[width=\linewidth]{pictures/gnuplot/3d/Speed/production/SpeedAuslenkung.eps}
			\put(60,9){Geschwindigkeit [cm/s]}
			\put(12,14){Mr [-]}
			\put(1,58){Auslenkung [cm]}
		\end{overpic}
		\caption{Geschwindigkeit, Mr und Auslenkung}
		\label{fig:SpeedAuslenkung}
	\end{center}
\end{figure}

\subsection{Impaktorradius}

Im Folgenden wird der Radius des Impaktors betrachtet. Nach Gleichung~\ref{form:Radius} wird der Radius der Kugel erhöht.

\begin{equation}
	0.5 [cm]\leq r_{g} \leq 10 [cm], \; \; \mbox{Schrittweite:} \; \Delta r_{g} = 0.1 [cm]
	\label{form:Radius}
\end{equation}

In Abbildung~\ref{fig:Radius} ist zu erkennen, dass der Radius nur einen geringen Einfluss auf die Anzahl der Aufschläge hat.\\
Bildet man die Daten in 3D ab, erkennt man leicht, dass die Anzahl der Aufschläge sich nur mit $Mr$ verändert. \\

\begin{figure}[h!]
	\begin{center}
		\begin{overpic}[width=\linewidth]{pictures/gnuplot/3d/Radius/production/Radius.eps}
			\put(44,4){Radius [cm]}
			\put(18,30){\rotatebox{90}{Mr [-]}}
			\put(83,60){Aufschläge [-]}
		\end{overpic}
		\caption{Radius, Mr und Anzahl der Aufschläge}
		\label{fig:Radius}
	\end{center}
\end{figure}

Ähnliche Ergebnisse werden auch bei der Auslenkung erreicht. Wie in Abbildung~\ref{fig:RadiusAuslenkung} zu sehen ist, nimmt die Auslenkung nur bei hohen $Mr$ auch mit durch Erhöhung des Radius zu. Im niederen $Mr$-Bereich sind die Auslenkungen bei $r_{g} = 0.5 [cm]$ und bei $r_{g} = 10.0 [cm]$ fast identisch bei $w = 0.04 [cm]$. Bei $Mr_{max} = 2.5$ ist die Auslenkung jedoch nur auf der zweiten Nachkommestelle größer, mit $w(r_{g}=0.5[cm]) = 1.27 [cm]$ und $w(r_{g}=10.0[cm]) = 1.29 [cm]$. 

\begin{figure}[h!]
	\begin{center}
		\begin{overpic}[width=\linewidth]{pictures/gnuplot/3d/Radius/production/RadiusAuslenkung.eps}
			\put(62,10){Radius [cm]}
			\put(14,13){Mr [-]}
			\put(1,56){Auslenkung [cm]}
		\end{overpic}
		\caption{Radius, Mr und Auslenkung}
		\label{fig:RadiusAuslenkung}
	\end{center}
\end{figure}

Auch wenn man die Kraft betrachtet, lassen sich ähnliche Trends erkennen. In Abbildung~\ref{fig:RadiusKraft} verändert sich die resultierende maximale Kraft im niederen $Mr$-Bereich nur minimal.\\
Die geringste Kraft tritt bei $r_{g,min} = 0.5 [cm]$ und $Mr_{min} = 0.01$ mit $P_{min} = 16.39 [kN]$ auf, während die Kraft bei gleichem $Mr$ und $r_{g,max} = 10.0 [cm]$ $P = 22.76 [kN]$ beträgt.\\
Analog zu der Auslenkung ergibt sich die signifikanteste Kraftänderung bei $Mr_{max}$. Die Kraft steigt von $P(r_{g,min}) = 136.50 [kN]$ auf $P(r_{g,max}) = 163.05 [kN]$ um den Faktor $1.194$ an. Verglichen mit der Kraftänderung von $Mr_{min}$ zu $Mr_{max}$, die bei $r_{g,max}$ mit dem Faktor $7.16$ verbunden ist, ist $\Delta P$ über die Radiusänderung eher gering. 

\begin{figure}[H]
	\begin{center}
		\begin{overpic}[width=\linewidth]{pictures/gnuplot/3d/Radius/production/RadiusKraft.eps}
			\put(45,4){Radius [cm]}
			\put(18,30){\rotatebox{90}{Mr [-]}}
			\put(80,62){Kraft [N]}
		\end{overpic}
		\caption{Radius, Mr und Kraft}
		\label{fig:RadiusKraft}
	\end{center}
\end{figure}


	%Im zweiten Teil des Hauptteils folgt die Darstellung der eigenen Leistung. Aufbauend auf den im vorigen Kapitel erarbeiteten Grundlagen  wird nun der Lösungsweg aufgezeigt. Dabei wird das prinzipielle Vorgehen zum Erreichen der Zielsetzung unter Einbindung der verwendeten Hilfsmittel (Maschinen, Geräte, Programme etc. inklusive der verwendeten Einstellungen) aufgezeigt. Dabei müssen sämtliche verwendeten Daten ersichtlich sein, sodass es jederzeit möglich ist, die ermittelten Ergebnisse zu reproduzieren. Zum Schluss erfolgt unter Berücksichtigung der Randbedingungen eine kritische Analyse der Ergebnisse mit möglichen Unsicherheiten und Fehlern. Aus der Diskussion der Ergebnisse (z.B. Vergleich von Messwerten und theoretischen Vorhersagen), wird schließlich der Nutzen erörtert und mögliche weiterführende Fragestellungen werden erarbeitet.
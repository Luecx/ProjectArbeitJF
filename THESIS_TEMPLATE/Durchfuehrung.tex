\chapter{Durchführung}
\label{chap:Durchfuehrung}

In der bereits hergeleiteten Theorie wird der Lösungsweg für das beschrieben Problem vorgestellt. Mit dem zur Verfügung gestelltem Skript, das auf dieser Theorie basiert wird nun die Parameterstudie durchgeführt. Um relevante Ergebnisse zu erlangen, müssen zuerst weitere Gleichungen in das Skript eingearbeitet werden.\\
In dem Skript wird eine Konstante $c$ verwendet, die aus der Hertz'schen Pressung kommt. Um die Berechnungsmöglichkeiten zu erweitern, wird $c$ nun durch eine Gleichung der Hertz'schen Pressung ersetzt, die wie folgt Hergeleitet wird:\\
Die Gleichung für die Hertz'sche Pressung eines Kugel-Kugel-Punktkontaktes lautet wie in Gleichung~\ref{form:ZvonP}

\begin{equation}
	\label{form:ZvonP}
	z(P) = \left[ \frac{9 P^{2} (1 - \nu^{2})^{2}}{2 E^{2}} \cdot \left( \frac{1}{2 r_{1}} + \frac{1}{2 r_{2}} \right) \right]^{\frac{1}{3}}
\end{equation}

Da einer der beiden Körper eine Platte mit dem Radius $r_{p} = \infty$ ist, folgt durch einsetzen von $r_{p}$ und umstellen nach P:

\begin{equation}
	\label{form:PvonZ}
	P(z) = \sqrt{\frac{4 r_{g}}{9}} \cdot \frac{E}{(1 - \nu^{2})} \cdot z^{\frac{3}{2}}
\end{equation}
	
Hier ist $\frac{E}{(1 - \nu^{2})}$ der reduzierte E-Modul, welcher durch Gleichung~\ref{form:redE} beschrieben wird.

\begin{equation}
	\label{form:redE}
	\frac{E}{(1 - \nu^{2})} = \left( \frac{(1 - \nu_{p}^{2})}{E_{p}} + \frac{(1 - \nu_{g}^{2})}{E_{g}} \right)^{-1}
\end{equation} 

Nun wird $c$ als Vorfaktor von $z^{\frac{3}{2}}$ in Gleichung~\ref{form:PvonZ} definiert, sodass folgt:

\begin{equation}
	\label{form:c}
	c = \sqrt{\frac{4 r_{g}}{9}} \cdot \frac{E}{(1 - \nu^{2})}
\end{equation}

Im Skript wurden Gleichungen~\ref{form:redE} und~\ref{form:c} unter den Konstanten implementiert und die einstellbaren Parameter der Platte und der Impaktor respektive um $E_{p}$ und $\nu_{p}$, bzw. $E_{g}$ und $\nu_{g}$ erweitert.\\

\section{Parameterstudie}

Für die Studie wurden folgende Annahmen getroffen: 

\begin{enumerate}
	\item{Die Platte und der Impaktor sind isotrop, mit konstanten Stoffwerten}
	\item{Der Radius des Impaktors bleibt konstant, die Masse kann aber verändert werden. (Siehe Abbildung~\ref{})}
	\item{Die Platte und der Impaktor sind aus identischem Material, mit gleichem E-Modul und Poissonzahl $\nu$}
\end{enumerate}




	%Im zweiten Teil des Hauptteils folgt die Darstellung der eigenen Leistung. Aufbauend auf den im vorigen Kapitel erarbeiteten Grundlagen  wird nun der Lösungsweg aufgezeigt. Dabei wird das prinzipielle Vorgehen zum Erreichen der Zielsetzung unter Einbindung der verwendeten Hilfsmittel (Maschinen, Geräte, Programme etc. inklusive der verwendeten Einstellungen) aufgezeigt. Dabei müssen sämtliche verwendeten Daten ersichtlich sein, sodass es jederzeit möglich ist, die ermittelten Ergebnisse zu reproduzieren. Zum Schluss erfolgt unter Berücksichtigung der Randbedingungen eine kritische Analyse der Ergebnisse mit möglichen Unsicherheiten und Fehlern. Aus der Diskussion der Ergebnisse (z.B. Vergleich von Messwerten und theoretischen Vorhersagen), wird schließlich der Nutzen erörtert und mögliche weiterführende Fragestellungen werden erarbeitet.
\chapter{Durchführung (eigene Leistung)}
\label{chap:Durchfuehrung}
Im zweiten Teil des Hauptteils folgt die Darstellung der eigenen Leistung. Aufbauend auf den im vorigen Kapitel erarbeiteten Grundlagen  wird nun der Lösungsweg aufgezeigt. Dabei wird das prinzipielle Vorgehen zum Erreichen der Zielsetzung unter Einbindung der verwendeten Hilfsmittel (Maschinen, Geräte, Programme etc. inklusive der verwendeten Einstellungen) aufgezeigt. Dabei müssen sämtliche verwendeten Daten ersichtlich sein, sodass es jederzeit möglich ist, die ermittelten Ergebnisse zu reproduzieren. Zum Schluss erfolgt unter Berücksichtigung der Randbedingungen eine kritische Analyse der Ergebnisse mit möglichen Unsicherheiten und Fehlern. Aus der Diskussion der Ergebnisse (z.B. Vergleich von Messwerten und theoretischen Vorhersagen), wird schließlich der Nutzen erörtert und mögliche weiterführende Fragestellungen werden erarbeitet.